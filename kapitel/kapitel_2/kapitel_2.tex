\newpage


\section{Methodology}\label{sec:methodology}

\subsection{Prototyping}\label{subsec:prototyping}

\subsubsection{Introduction to the Prototyping Methodology According to Floyd}\label{subsubsec:ptintro}

The prototyping approach developed by Christiane Floyd represents a structured methodology used primarily in software
development to improve communication between developers and users, reduce misunderstandings, and ultimately enhance the
quality of the final product.\autocite[cf.][p.2-3]{floydSystematicLookPrototyping1984} This methodology provides an
alternative to the traditional linear, phase-oriented development process by introducing a dynamic element of iteration
and feedback.\autocite[cf.][p.3]{floydSystematicLookPrototyping1984} As a result, it facilitates a more
interactive and user-centered development process.\autocite[cf.][p.3-4]{floydSystematicLookPrototyping1984}

Floyd’s prototyping process is structured as a cyclical sequence of four distinct steps: functional selection,
construction, evaluation, and further use.\\ The first step, Functional Selection, involves identifying the specific
functions that the prototype should demonstrate. The selected functionalities are derived from the relevant work tasks
to ensure meaningful demonstrations, while acknowledging that the prototype does not need to represent the final product
comprehensively. This allows for a degree of flexibility in the selection and prioritization of features to be included
in the prototype.\autocite[cf.][p.4]{floydSystematicLookPrototyping1984}

The second step, Construction, entails building the prototype using techniques and tools that enable rapid development
and easy adjustments. At this stage, the focus is not on developing a fully functional system, but rather on
demonstrating and assessing specific aspects of the final product. This approach enables the prototype to act as a tool
for exploring different solutions and gathering feedback.\autocite[cf.][p.4]{floydSystematicLookPrototyping1984}

Evaluation, the third step in the process, serves as the cornerstone of the prototyping methodology. In this step,
feedback from all relevant stakeholders—including potential users—is collected and analyzed to refine and guide
subsequent development stages. This iterative process ensures that the prototype evolves in alignment with user
expectations and needs.\autocite[cf.][p.4-5]{floydSystematicLookPrototyping1984}

The final step, Further Use, determines the prototype's role after the evaluation phase. Depending on its effectiveness
and the degree to which it meets requirements, the prototype can be discarded, modified for continued use, or serve as a
foundation for the final product. This flexibility is critical in accommodating evolving requirements and objectives,
making the prototyping approach particularly valuable in contexts where specifications are expected to change
frequently.\autocite[cf.][p.5]{floydSystematicLookPrototyping1984}

Floyd’s methodology also categorizes prototyping into three primary approaches, each based on the goals and context of
development: exploratory prototyping, experimental prototyping, and evolutionary prototyping.
\autocite[cf.][p.6]{floydSystematicLookPrototyping1984}\\ Exploratory Prototyping is
primarily used to clarify requirements and foster creative cooperation between developers and users during the early
stages of development. It is particularly useful when there is a lack of clarity on the system’s final objectives, as it
allows for broad experimentation and refinement of ideas before committing to a specific solution.
\autocite[cf.][p.6-7]{floydSystematicLookPrototyping1984}

In contrast, Experimental Prototyping focuses on testing proposed solutions to validate specific hypotheses, such as
user interface design, system performance, or algorithm feasibility. This approach might involve techniques such as full
functional simulation or human interface simulation to verify that the proposed design meets the intended objectives.
\autocite[cf.][p.8-10]{floydSystematicLookPrototyping1984}

Finally, Evolutionary Prototyping treats the prototype as a system that continuously evolves to adapt to changing
requirements over time. Each version of the prototype serves as a basis for the next iteration, incorporating new
insights and user feedback. This approach is especially beneficial in scenarios where requirements are expected to
change frequently, rendering a static set of requirements impractical.
\autocite[cf.][p.10-12]{floydSystematicLookPrototyping1984}

To support these prototyping processes, various techniques and tools can be utilized.
\autocite[cf.][p.12]{floydSystematicLookPrototyping1984} \\Modular Design, for instance, encourages the use of small,
independent modules that can be replaced or refined as needed, thereby facilitating iterative development and easing
integration into the final product.\autocite[cf.][p.12]{floydSystematicLookPrototyping1984} \\Additionally, Dialogue
Design plays a crucial role in ensuring that the user interface is adaptable and transparent, which enables effective
evaluation and modification of the user experience.\autocite[cf.][p.12]{floydSystematicLookPrototyping1984}
\\Furthermore, Simulation techniques allow for simulating aspects of the final system that are not yet fully
implemented, enabling the assessment of system performance and user interaction without the need for a complete
implementation.\autocite[cf.][p.13]{floydSystematicLookPrototyping1984}

\subsubsection{Selection of Prototyping Approach}\label{subsubsec:ptselection}

Given the context and objectives of this project, the decision was made to adopt the experimental prototyping approach.
This choice is rooted in the fact that the project requirements have already been well-defined through a comprehensive
requirements engineering process, thus eliminating the need for exploratory prototyping. The clear specification of
functionalities and user expectations ensures that the focus can shift from understanding requirements to validating and
testing specific design choices.

Furthermore, the experimental approach is particularly well-suited for scenarios where a prototype is intended to serve
as a preliminary proof of concept rather than a foundation for incremental development. This aligns perfectly with the
anticipated lifecycle of the prototype in this project, which is expected to be discarded once the MVP phase begins. As
the project moves towards the MVP stage, a fresh start will be made, incorporating only validated concepts and findings
from the experimental prototype. Therefore, evolutionary prototyping is not applicable, as it is primarily designed for
projects that involve iterative refinement and continuous evolution of the same prototype.

The experimental prototyping approach allows for focused experimentation with various design elements, interface
interactions, and technical implementations, all within a controlled environment that does not necessitate long-term
integration into the final product.

\subsubsection{Application of Experimental Prototyping in the Project}\label{subsubsec:ptapplication}

The experimental prototyping approach will be implemented in this project with a focus on validating user interface
designs, interactions, and core functionalities against predefined requirements. The development of the prototype will
leverage a set of carefully selected techniques that facilitate rapid iteration and feedback.

A key technique employed is simulation, which is used to mimic certain system behaviors without integrating the
prototype into live production environments. This decision is motivated by the limited scope of the prototype and the
intention to avoid disruptions to existing systems. By relying on test data instead of actual production data, the
prototype can simulate real-world scenarios and provide valuable insights into its performance and user experience.

Moreover, the prototype will not attempt to implement every function in its final depth and breadth. Instead, certain
features will be simulated to convey the look and feel of the system, providing a realistic representation of how the
final product would function. This is where Modular Design plays a crucial role. By leveraging modular design principles
, the prototype can separate the user interface from the underlying logic and backend functionalities. This allows
specific modules to be developed exclusively for the UI, simulating the presence of certain features without the need
for fully developed backend logic. For instance, UI elements such as buttons, forms, and interactive components can be
displayed and interacted with as if they were functional, even though the backend processing is either simulated or
entirely absent. This approach enables the developer to receive early feedback on key aspects of the
design and functionality without committing extensive resources to full-scale implementation.

By focusing on these aspects, the prototype will serve as a learning tool, guiding the refinement of requirements and
design choices before moving into the more resource-intensive MVP phase.
