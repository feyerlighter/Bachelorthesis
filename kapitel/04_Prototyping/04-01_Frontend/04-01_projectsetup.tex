\paragraph{Project Setup}%main.ts, App.vue, index.ts, index.vue

The project setup involves the initialization and configuration of the core Vue.js application, integrating essential
plugins and defining the main structure of the application. This setup includes several key files: \texttt{main.ts},
\texttt{App.vue}, \texttt{index.ts} (in the plugins folder), and \texttt{index.vue}. These files work together to create
a consistent and well-organized foundation for the chatbot system.

The \texttt{main.ts}
file is responsible for bootstrapping the Vue.js application, registering essential plugins, and mounting the root
component. This entry point initializes the app with Vue.js and applies global configurations.

Code \ref{lst:main-ts-initialization} demonstrates how the application is initialized and mounted.

% @formatter:off
\begin{lstlisting}[language=JavaScript, caption={Bootstrapping the Application (\texttt{main.ts})},
  firstnumber=6,label={lst:main-ts-initialization}]
import './styles/main.scss'

// Plugins
import { registerPlugins } from '@/plugins'
import { VueQueryPlugin } from '@tanstack/vue-query'

// Components
import App from './App.vue'

// Composables
import { createApp } from 'vue'

const app = createApp(App)

registerPlugins(app)
app.use(VueQueryPlugin)

app.mount('#app')
\end{lstlisting}
% @formatter:on

The project imports a global stylesheet (\texttt{main.scss}) to ensure consistent styling across
all components. The \texttt{registerPlugins} function, imported from the \texttt{plugins} folder, sets up
essential plugins such as Vuetify and Vue Router. The \texttt{VueQueryPlugin} plugin is used for data fetching and
caching, improving the app’s efficiency and reactivity in managing API data. Finally, the root component
(\texttt{App.vue}) is mounted to the \texttt{\#app} element in the HTML, starting the application.

The \texttt{App.vue} file defines the overall layout of the application using Vuetify’s layout components.
It serves as the root component that houses all other views and components.

The template in Code \ref{lst:app-vue-layout} illustrates how Vuetify components are used to create a structured layout.

% @formatter:off
\begin{lstlisting}[language=HTML, caption={Setting Up the Root Layout (\texttt{App.vue})},
  firstnumber=1,label={lst:app-vue-layout}]
<template>
  <v-app>
    <v-main>
      <router-view />
    </v-main>
  </v-app>
</template>
\end{lstlisting}
% @formatter:on

Vuetify’s \texttt{v-app} and \texttt{v-main} components create a structured layout
, ensuring that the application adheres to material design principles.
The \texttt{<router-view />} dynamically renders the current route’s component based on the Vue Router configuration.

The \texttt{plugins/index.ts} file contains the \texttt{registerPlugins} function, which is used to
register essential plugins such as Vuetify and Vue Router with
the Vue.js application. This function centralizes plugin configuration, making it easier to manage.

The following code snippet shows how plugins are registered using
\texttt{registerPlugins}, as shown in Code \ref{lst:plugins-registration}.

% @formatter:off
\begin{lstlisting}[language=JavaScript, caption={Registering Essential Plugins (\texttt{plugins/index.ts})},
  firstnumber=8,label={lst:plugins-registration}]
import vuetify from './vuetify'
import router from '../router'

// Types
import type { App } from 'vue'

export function registerPlugins(app: App) {
  app.use(vuetify).use(router)
}
\end{lstlisting}
% @formatter:on

The function takesa Vue app instance as an argument and applies the necessary plugins using \texttt{app.use()}.
This pattern keeps the plugin registration clean and consistent across the application.

The \texttt{index.vue} file serves as a key layout component that defines the main structure
and user interface elements of the application. It includes the navigation bar, app bar, and the main container
where chat functionality and other key features are rendered.

The app bar at the top of the page, shown in Code \ref{lst:app-bar}, provides access to the sidebar, theme switcher, and
user login dialog.

% @formatter:off
\begin{lstlisting}[language=HTML, caption={Setting Up the App Bar (\texttt{index.vue})},
  firstnumber=17,label={lst:app-bar}]
<v-app-bar prominent :elevation="0" color="background" class="pr-3">
  <template #prepend>
    <v-btn icon="$sidebar" size="small" color="secondary" variant="text" @click.stop="drawer = !drawer" />
  </template>

  <v-toolbar-title>
    <span class="text-page-header-color bg-logo-background text-center pa-2 mr-4">
      <v-icon icon="$telekom" size="xs" class="my-2" />
    </span>
    <span class="text-page-header-color font-weight-bold">Digital Chief Procurement Officer</span>
  </v-toolbar-title>

  <v-spacer></v-spacer>
  <div class="d-flex align-center">
    <ThemeSwitcher />
    <UserLoginDialog v-model="showUserLoginDialog" :persistend="showUserLoginDialogPersistend" />
  </div>
</v-app-bar>
\end{lstlisting}
% @formatter:on

The app bar includes a sidebar toggle, the project title, and user-related features such as the theme switcher
and login dialog.

The main container divides the page into two sections: one for the chat and the other for displaying the checklist,
service description, and shopping cart. This layout provides an intuitive and organized interface, as shown in Code
\ref{lst:main-container}.

% @formatter:off
\begin{lstlisting}[language=HTML, caption={Main Container Layout (\texttt{index.vue})},
  firstnumber=49,label={lst:main-container}]
<v-main>
  <div class="main-container d-flex justify-between ga-3 w-100 no-wrap">
    <div class="w-50 rounded-lg" :class="{ 'ml-3': drawer }">
      <Chat />
    </div>
    <div class="w-50 rounded-lg d-flex flex-column ga-3">
      <Checklist />
      <ServiceDescription />
      <ShoppingCart />
    </div>
  </div>
</v-main>
\end{lstlisting}
% @formatter:on

This layout ensures that the chat remains prominent, while additional information is organized on the right side,
enhancing the user’s interaction experience.
