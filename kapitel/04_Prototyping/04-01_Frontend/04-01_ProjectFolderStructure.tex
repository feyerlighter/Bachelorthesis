\paragraph{Project Folder Structure}

The project structure is organized to maintain a clear separation of concerns, making it scalable and maintainable. Its
structure is shown in Code \ref{lst:project-structure}. The root of the \texttt{src}
directory contains folders and files that serve specific purposes within the application.

% @formatter:off
\begin{lstlisting}[language=bash, caption={Project Structure}, label={lst:project-structure}]
src/
│
├── api/                  # API schemas and client utilities
│
├── assets/               # Static assets (images, fonts, etc.)
│
├── components/           # Reusable Vue components
│
├── composables/          # Reusable Vue composables for logic
│
├── pages/                # Page-specific Vue components
│
├── plugins/              # Plugin registrations and configurations
│
├── router/               # Vue Router configuration
│
├── stores/               # Global state management (using reactive stores)
│
├── styles/               # Global SCSS files for styling
│
├── types/                # TypeScript type definitions
│
├── App.vue               # Root Vue component defining the app layout
├── main.ts               # Application entry point for Vue initialization
└── vite-env.d.ts         # Type definitions for Vite
\end{lstlisting}
% @formatter:on

The \texttt{api}
folder contains files defining API schemas and utility functions to interact with backend services. It holds definitions
related to API paths and types, facilitating communication between frontend and backend. The \texttt{assets}
folder stores static files like images, fonts, and other media assets used in the project. The \texttt{components}
folder contains reusable Vue components that are used across various parts of the application, including UI elements
like buttons, dialogs, and cards. The \texttt{composables}
folder includes reusable logic in the form of Vue composables, which encapsulate functionalities such as API handling
and state management.

The \texttt{pages}
folder holds Vue components representing individual pages or views of the application, such as home, login, or dashboard
pages. The \texttt{plugins}
folder registers and configures external plugins like Vuetify, Vue Router, and others used throughout the project. The
\texttt{router}
folder contains Vue Router configuration files, which define the routing rules and paths within the application. The
\texttt{stores}
folder manages global application state using Vue’s reactivity system, centralizing state that needs to be shared across
components.

The \texttt{styles}
folder contains global SCSS files for styling the application consistently, including variables, mixins, and common
styles. The \texttt{types}
folder stores TypeScript type definitions used across the application to enforce consistent data structures. At the root
of the \texttt{src} folder, \texttt{App.vue} is the root Vue component that defines the application’s main layout. The
\texttt{main.ts}
file serves as the main entry point of the application where Vue is initialized, plugins are registered, and the root
component is mounted. Lastly, \texttt{vite-env.d.ts}
defines types and environment variables for the Vite development environment.
