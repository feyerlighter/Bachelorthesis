\paragraph{User Login and Session Handling}%UserLoginDialog.vue

This section focuses on the user login and session management, enabling users to authenticate themselves by entering a
User \ac{ID}. The system leverages Vue.js components, Vuetify elements, and reactive state management to build an
intuitive login dialog. This feature ensures that only authenticated users can interact with the chatbot, thus
personalizing the user experience and maintaining the session across multiple interactions.

The login component is implemented using \ac{TS}, providing type safety and enhanced tooling support, which is
beneficial for large-scale projects. Below is a breakdown of the code's key functionalities.

The following code snippet imports essential modules and components used throughout the login process. See
Code \ref{lst:importing-dependencies}.

\begin{lstlisting}[language=JavaScript, caption={Importing Dependencies (\texttt{UserLoginDialog.vue})},
firstnumber=2,label={lst:importing-dependencies}]
import { userStore } from '@/stores/user-store'
import { onUpdated, ref, useTemplateRef } from 'vue'
import { VTextField } from 'vuetify/components/VTextField'
\end{lstlisting}

The \texttt{userStore}
is imported from a centralized state management store to handle user-specific data, such as the user \acs{ID}. The
\texttt{onUpdated}, \texttt{ref}, and \texttt{useTemplateRef} functions are imported from Vue, essential for reactive
programming, where component properties update automatically when the state changes. The \texttt{VTextField} from
Vuetify is used to render the input field in the login form.

Reactive references and validation rules are established in Code \ref{lst:reactive-variables-validation} to handle user
input.

\begin{lstlisting}[language=JavaScript, caption={Reactive Variables and Validation (\texttt{UserLoginDialog.vue})},
firstnumber=6,label={lst:reactive-variables-validation}]
const model = defineModel<boolean>()
defineProps<{ persistend: boolean }>()

const userIdInput = useTemplateRef<VTextField>('userIdInput')
const userId = ref<string>()
const userIdRules = ref<((v: string) => string | boolean)[]>([
(v: string) => (!!v && !!v.trim()) || 'User ID is required'
])
\end{lstlisting}

The \texttt{model} defines a reactive model to control the visibility of the login dialog, while the \texttt{persistend}
prop determines whether the dialog can be closed without user interaction. The \texttt{userId}
is a reactive reference holding the user input, while \texttt{userIdRules}
ensures that the input is valid, requiring a non-empty string to maintain data integrity.

The \texttt{onUpdated} lifecycle hook, shown in Code \ref{lst:synchronizing-lifecycle-hooks}, ensures the synchronization
of the user \acs{ID} with the session state stored in the \texttt{userStore}.

\begin{lstlisting}[language=JavaScript, caption={Synchronizing State with Lifecycle Hooks (\texttt{UserLoginDialog.vue})},
firstnumber=15,label={lst:synchronizing-lifecycle-hooks}]
onUpdated(() => {
userId.value = userStore.userId
})
\end{lstlisting}

This ensures that the correct user \acs{ID} is displayed in the form if it has been previously set.

Code \ref{lst:handling-user-login} handles the user login process.

\begin{lstlisting}[language=JavaScript, caption={Handling User Login (\texttt{UserLoginDialog.vue})},
firstnumber=19,label={lst:handling-user-login}]
function login() {
if (!userId.value) return

userStore.login(userId.value)
model.value = false
}
\end{lstlisting}

The \texttt{login} function first checks if the \texttt{userId} field is populated. If so, the
\texttt{userStore.login()} function is called to store the user \acs{ID} for future interactions. After a successful login,
the login dialog is closed, providing a seamless user experience.

The \ac{HTML} template uses Vuetify components to create the login interface, binding state variables for reactivity and
ensuring user input validation.

The login dialog is defined as follows in Code \ref{lst:login-dialog}.

\begin{lstlisting}[language=JavaScript, caption={Login Dialog (\texttt{UserLoginDialog.vue})},
firstnumber=28,label={lst:login-dialog}]
<v-dialog v-model="model" max-width="400" :persistent="persistend">
<template #activator="{ props: activatorProps }">
<v-btn v-bind="activatorProps" variant="text" class="text-none">
<div v-if="userStore.userId">{{ userStore.userId }}</div>
<div v-else>Login</div>
<template #append>
<v-icon v-if="userStore.userId">mdi-account-check</v-icon>
<v-icon v-else>mdi-account-plus</v-icon>
</template>
</v-btn>
</template>
</v-dialog>
\end{lstlisting}

The \texttt{v-model} directive links the dialog's visibility to the \texttt{model}
reactive variable. If the user is logged in (i.e., \texttt{userStore.userId}
is set), the button displays the user \acs{ID}; otherwise, it prompts the user to log in. Icons are dynamically displayed to
indicate the login state.

The input field for the user \acs{ID} is defined as shwon in Code \ref{lst:user-id-input}.

\begin{lstlisting}[language=JavaScript, caption={User ID Input Field (\texttt{UserLoginDialog.vue})},
firstnumber=57,label={lst:user-id-input}]
<v-text-field
ref="userIdInput"
v-model="userId"
:counter="10"
:rules="userIdRules"
label="User ID"
@keyup.enter="login"
clearable
></v-text-field>
\end{lstlisting}

The \texttt{v-text-field} is bound to the \texttt{userId} reactive reference, with a 10-character limit enforced by the
\texttt{counter} attribute. Validation rules are applied via \texttt{userIdRules}
, ensuring that users cannot submit invalid \acsp{ID}. The \texttt{@keyup.enter="login"}
directive enables users to press "Enter" to submit the form, simplifying interaction.

The login button is implemented as follows. See Code \ref{lst:login-button}.

\begin{lstlisting}[language=JavaScript, caption={Login Button (\texttt{UserLoginDialog.vue})},
firstnumber=70,label={lst:login-button}]
<v-btn @click="login" :disabled="!userId"> Login </v-btn>
\end{lstlisting}

The button is enabled only when a valid \texttt{userId}
is present, preventing users from submitting the form without valid input.

User login and session handling are critical for managing how users access the chatbot system. It ensures that only
authorized users can interact with the chatbot, which is essential for both security and personalization. Vuetify
components provide a professional and responsive interface, while Vue’s reactive system handles the dynamic nature of
login states, ensuring a smooth user experience without unnecessary page reloads or delays.