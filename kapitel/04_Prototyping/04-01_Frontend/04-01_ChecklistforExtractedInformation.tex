\paragraph{Checklist for Extracted Information}%

This section focuses on rendering a dynamic checklist that automatically updates based on the attributes extracted from
the chatbot conversation. The checklist helps ensure that all required information is provided, enhancing the user
experience by visually tracking the progress of filling out necessary details. The three components involved—
\texttt{Checklist.vue}, \texttt{Checkbox.vue}, and \texttt{ChecklistItem.vue}
—work together to display attributes as checklist items and indicate whether they have been completed.

The \texttt{Checklist.vue}
component is responsible for rendering the overall checklist, which displays each attribute as a separate checklist
item. The checklist is populated based on the state of the current case, dynamically fetched from the
\texttt{selectedCaseStore}.

The following snippet demonstrates how the attributes are computed based on the case state. Only attributes that are not
related to the "service description" are included in the checklist, as shown in Code
\ref{lst:checklist-computed-attributes}.

% @formatter:off
\begin{lstlisting}[language=JavaScript, caption={Computing Attributes for Checklist (\texttt{Checklist.vue})},
  firstnumber=6,label={lst:checklist-computed-attributes}]
const attributes = computed(() =>
  selectedCaseStore.case.state
    ?.flatMap((s) => s.attributes)
    .filter((a) => a.attribute_id !== 'service_description')
)
\end{lstlisting}
% @formatter:on

This code processes the case's state to extract relevant attributes for display as checklist items, filtering
out irrelevant attributes like "service description." The computed
property ensures that the checklist dynamically updates as new information is added during the conversation.

The template, shown in Code \ref{lst:checklist-rendering}, shows how the checklist items are rendered, using the
\texttt{ChecklistItem} component for each attribute.

% @formatter:off
\begin{lstlisting}[language=HTML, caption={Rendering the Checklist Items (\texttt{Checklist.vue})},
  firstnumber=20,label={lst:checklist-rendering}]
<div v-if="attributes" class="overflow-y-auto d-flex flex-column ga-3">
  <ChecklistItem
    v-for="attribute in attributes"
    :key="attribute.attribute_id"
    v-bind="attribute"
  />
</div>
\end{lstlisting}
% @formatter:on

Each attribute is rendered using the \texttt{ChecklistItem} component,
representing an individual item in the checklist. If no attributes
are available, a placeholder message prompts the user to continue the conversation to provide the necessary information.

The \texttt{Checkbox.vue} component represents a simple visual checkbox indicating
whether a checklist item has been completed. It uses Vuetify's \texttt{v-icon} to display a checkmark when the
item is marked as "checked."

Code \ref{lst:checkbox-state} shows how the checkbox state is managed via the \texttt{isChecked} prop.

% @formatter:off
\begin{lstlisting}[language=JavaScript, caption={Managing the Checkbox State (\texttt{Checkbox.vue})},
  firstnumber=2,label={lst:checkbox-state}]
const { isChecked = false } = defineProps<{
  isChecked: boolean
}>()
\end{lstlisting}
% @formatter:on

This allows the checkbox to display its state (checked or unchecked) based on the value passed from the parent
component.

The template below in Code \ref{lst:checkbox-template} renders the checkbox, which changes its appearance depending on
whether it is checked.

% @formatter:off
\begin{lstlisting}[language=HTML, caption={Rendering the Checkbox (\texttt{Checkbox.vue})},
  firstnumber=8,label={lst:checkbox-template}]
<div class="checkbox" :class="{ checked: isChecked }">
  <v-icon class="check-icon" size="35" color="success" icon="$check" />
</div>
\end{lstlisting}
% @formatter:on

When \texttt{isChecked} is true, the \texttt{check-icon} is displayed, visually indicating
that the attribute has been completed.

The \texttt{ChecklistItem.vue} component is responsible for rendering each individual item in
the checklist. It receives an attribute as a prop and checks its value to determine whether the item is completed
and what tags or values should be displayed.

The following snippet defines the props and computes the tags for each
item, as demonstrated in \ref{lst:checklistitem-tags}.

% @formatter:off
\begin{lstlisting}[language=JavaScript, caption={Computing Tags for Checklist Item (\texttt{ChecklistItem.vue})},
  firstnumber=16,label={lst:checklistitem-tags}]
const tags = computed<string[]>(() => (value ? [value].flat().map((x) => String(x)) : []))
\end{lstlisting}
% @formatter:on

The \texttt{tags} property is computed from the attribute’s value
, allowing the component to display relevant tags or information associated with the item. The \texttt{.flat()
} method ensures that arrays are handled correctly, and \texttt{String(x)} converts each value to a string for display.

The template below renders the checklist item, including the checkbox and any associated tags
, as shown in \ref{lst:checklistitem-template}.

% @formatter:off
\begin{lstlisting}[language=HTML, caption={Rendering the Checklist Item (\texttt{ChecklistItem.vue})},
  firstnumber=21,label={lst:checklistitem-template}]
<div class="d-flex align-center ga-2">
  <Checkbox :is-checked="!!value" />
  <h4>{{ title }}</h4>
</div>
<div class="d-flex align-center ga-2">
  <div style="width: 34px; height: 100%"></div>
  <div v-if="tags.length" class="d-flex ga-2">
    <div v-for="tag in tags" :key="tag">
      <v-chip variant="flat" density="compact" color="chip-background" label>
        <b>{{ tag }}</b>
      </v-chip>
    </div>
  </div>
</div>
\end{lstlisting}
% @formatter:on

This section of the template displays a checkbox, the item title, and any tags associated
with the item. The \texttt{v-chip} component is used to display the tags in a compact format, making the information
visually accessible and easy to understand.
