\paragraph{Theme and User Settings Persistence}%theme-switcher

This section covers how the application persists the user's theme preferences (dark or light mode) across sessions,
ensuring a consistent user experience. The \texttt{ThemeSwitcher.vue}
component enables users to toggle between themes, with their choice saved using local storage. Vue’s reactive system and
Vuetify’s theme management are utilized to implement this functionality.

The user's selected theme is stored in the browser’s local storage, allowing the theme preference to persist even after
the user closes and reopens the application. The following code snippet demonstrates how local storage is utilized for
theme persistence, as shown in \ref{lst:theme-local-storage}.

% @formatter:off
\begin{lstlisting}[language=JavaScript, caption={Using Local Storage for Theme Persistence (\texttt{ThemeSwitcher.vue})},
  firstnumber=7,label={lst:theme-local-storage}]
const storedTheme = useLocalStorage('selected-theme', theme.global.name.value)
\end{lstlisting}
% @formatter:on

This line uses \texttt{useLocalStorage} from \texttt{@vueuse/core} to store the current theme
in local storage. The key \texttt{selected-theme} is used to save the theme's name. If no previous value
is found, the default theme (\texttt{theme.global.name.value}) is stored.

When the component is mounted, the stored theme is applied to ensure the user’s preference
is loaded and set correctly. This ensures that the selected theme from the last session is applied when the user
returns to the application, as demonstrated in \ref{lst:theme-sync-on-mount}.

% @formatter:off
\begin{lstlisting}[language=JavaScript, caption={Syncing Theme on Mount (\texttt{ThemeSwitcher.vue})},
  firstnumber=9,label={lst:theme-sync-on-mount}]
onMounted(() => (theme.global.name.value = storedTheme.value))
\end{lstlisting}
% @formatter:on

Thisline of code ensures that, as soon as the component is mounted, the theme stored in local storage
is applied to the application.

The component watches for any changes in the theme and updates the local storage accordingly
, ensuring that the latest theme choice is always saved. This is shown in \ref{lst:theme-watch}.

% @formatter:off
\begin{lstlisting}[language=JavaScript, caption={Watching for Theme Changes (\texttt{ThemeSwitcher.vue})},
  firstnumber=11,label={lst:theme-watch}]
watch(
  () => theme.global.name.value,
  (value) => (storedTheme.value = value)
)
\end{lstlisting}
% @formatter:on

The \texttt{watch} function observes changes to \texttt{theme.global.name.value} (
the current theme). Whenever the user switches themes, the new value is stored in \texttt{storedTheme
}, updating the local storage.

The template part of the component in Code \ref{lst:theme-template} includes a Vuetify \texttt{v-switch} that
allows users to toggle between light and dark themes.

% @formatter:off
\begin{lstlisting}[language=HTML, caption={Template for Theme Toggle (\texttt{ThemeSwitcher.vue})},
  firstnumber=18,label={lst:theme-template}]
<v-switch
  v-model="theme.global.name.value"
  false-value="telekom-light"
  true-value="telekom-dark"
  class="px-3"
  color="primary"
  hide-details
>
  <template #prepend>
    <v-icon icon="$darkMode" color="primary" size="small" />
    <span class="pl-4">Dark Mode</span>
  </template>
</v-switch>
\end{lstlisting}
% @formatter:on

The \texttt{v-switch} is bound to \texttt{theme.global.name.value}, allowing users
to switch between the \texttt{telekom-light} and \texttt{telekom-dark} themes. The current theme is dynamically updated
, and the \texttt{v-switch} provides a user-friendly toggle for light and dark modes.

All this ensures that the user’s theme preference is saved and restored across sessions
, enhancing the user experience by maintaining a consistent interface.
The combination of local storage and Vuetify’s theme management provides
a seamless and intuitive approach to theme persistence, ensuring
that the user’s selected theme is applied even after the application is reopened.
