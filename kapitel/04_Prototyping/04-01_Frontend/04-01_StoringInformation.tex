\paragraph{Storing Information}%folder stores

This section focuses on how the chatbot system stores and manages key pieces of data related to user sessions and
selected cases. Proper state management and persistence of information are crucial for maintaining a seamless user
experience, especially when handling interactions across multiple sessions or refreshing the application. Two key pieces
of state—the selected case and the user session—are managed using Vue's reactivity system. The state is stored locally,
allowing the application to maintain important information between page loads.

The \texttt{selected-case.store.ts}
file manages the state of the selected case throughout the chatbot session. It uses Vue's \texttt{reactive}
function to create a reactive store that can be shared across components. This is essential for keeping the selected
case in sync across different parts of the application.

The following code snippet shows how the \texttt{selectedCaseStore} is set up and managed, as demonstrated in
\ref{lst:selected-case-store}.

% @formatter:off
\begin{lstlisting}[language=JavaScript, caption={Setting up the Case Store (\texttt{selected-case.store.ts})},
  firstnumber=10,label={lst:selected-case-store}]
export const selectedCaseStore = reactive<CaseStore>({
  case: {
    id: '',
    created_at: ''
  },
  setCase(caseData: Case) {
    this.case = caseData
  },
  clearCase() {
    this.case = {
      id: '',
      created_at: ''
    }
  }
})
\end{lstlisting}
% @formatter:on

In this snippet, \texttt{selectedCaseStore}
is a reactive object that holds the current case state. This object provides essential methods to manage the state. The
\texttt{setCase}
method updates the case state with new data whenever a case is selected, ensuring that the correct data is displayed
across components when switching between different cases during a chat session. Additionally, the \texttt{clearCase}
method resets the case state, which is particularly useful when the user finishes working on a case or logs out.

By using a reactive store, the case state remains in sync across all components that rely on it, making the user
experience consistent and dynamic.

The \texttt{user-store.ts}
file handles the state of the user's session, including login and logout functionality. It uses Vue’s reactive system in
combination with the \texttt{useLocalStorage} utility from the \texttt{@vueuse/core} library to persist the user's
\texttt{userId} across sessions.

The following code snippet shows how the \texttt{userStore}
manages the user’s login and session data, as demonstrated in \ref{lst:user-store}.

% @formatter:off
\begin{lstlisting}[language=JavaScript, caption={Managing the User Session (\texttt{user-store.ts})},
  firstnumber=4,label={lst:user-store}]
const storedUserId = useLocalStorage<string | undefined>('user-id', undefined)

export const userStore: User = reactive<User>({
  userId: storedUserId.value,
  login(userId: string) {
    storedUserId.value = userId
    this.userId = userId
  },
  logout() {
    storedUserId.value = undefined
    this.userId = undefined
  }
})
\end{lstlisting}
% @formatter:on

This code leverages Vue’s reactive system in combination with the \texttt{useLocalStorage} utility from the
\texttt{@vueuse/core} library to ensure that the user’s
\texttt{userId} persists across sessions. The implementation begins
by initializing \texttt{storedUserId} from local storage, and this value is then used as the default for
\texttt{userId} within the reactive store.

When the \texttt{login} method is triggered, the user’s \texttt{userId} is stored both in the reactive
state and in local storage, allowing other components to reactively update based on the user’s login status. Conversely,
when the \texttt{logout} method is executed, both the reactive state and local storage are cleared,
ensuring that the session is properly ended when the user logs out.

This approach ensures that the user does not have to log in again after refreshing the page or
closing and reopening the application, thanks to the persistence of the \texttt{userId} in local storage.
