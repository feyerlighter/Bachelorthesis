\paragraph{Supplier Matching in Chat}

This section introduces the supplier selection functionality within the chat, allowing users to browse a list of
potential suppliers and select one based on criteria such as price and name. The feature is implemented through the
\texttt{SupplierGallery.vue} component, which displays supplier options and lets users select a supplier for further
interaction. This helps in narrowing down options for services or products that match the user’s needs, streamlining
the supplier matching process.

The following code defines the structure of a supplier and initializes a list of suppliers to be displayed in the chat,
as shown in Code \ref{lst:supplier-interface-data}.

\begin{lstlisting}[language=JavaScript, caption={Defining the Supplier Interface and Data
    (\texttt{SupplierGallery.vue})}, firstnumber=4,label={lst:supplier-interface-data}]
interface SupplierGalleryItem {
  id: string
  name: string
  image: string | undefined
  price: string
  url: string
}

const suppliers: SupplierGalleryItem[] = [
  {
    id: '1',
    name: 'Supplier 1',
    image: undefined,
    price: '125.000 - 150.000 €',
    url: 'https://www.google.com'
  },
  {
    id: '2',
    name: 'Supplier 2',
    image: 'https://via.placeholder.com/150',
    price: '150.000 - 200.000 €',
    url: 'https://www.google.com'
  },
  {
    id: '3',
    name: 'Supplier 3',
    image: 'https://via.placeholder.com/150',
    price: '200.000 - 250.000 €',
    url: 'https://www.google.com'
  }
]
\end{lstlisting}

The \texttt{SupplierGalleryItem} interface defines the structure for each supplier, including properties like
\texttt{id}, \texttt{name}, \texttt{image}, \texttt{price}, and \texttt{url}. The \texttt{suppliers} array holds the
data for three example suppliers, each with its own details. Some suppliers have images, while others do not, showing
flexibility in how supplier information is displayed.

The following function allows the user to select a supplier from the list, storing the selected supplier in the reactive
reference \texttt{selectedSupplier}, as demonstrated in Code \ref{lst:select-supplier-function}.

\begin{lstlisting}[language=JavaScript, caption={Selecting a Supplier (\texttt{SupplierGallery.vue})},
  firstnumber=36,label={lst:select-supplier-function}]
let selectedSupplier = ref<SupplierGalleryItem | undefined>(undefined)
function selectSupplier(supplier: SupplierGalleryItem) {
  selectedSupplier.value = supplier
}
\end{lstlisting}

The \texttt{selectedSupplier} reference keeps track of the currently selected supplier. The \texttt{selectSupplier}
function updates this reference when the user clicks on a supplier, ensuring that the selected supplier is highlighted
in the UI. This allows users to easily compare and select from different supplier options.

The template code below renders the list of suppliers as clickable cards. Each card displays the supplier’s name, price,
image (if available), and a button for selecting the supplier. The card structure is shown in Code
\ref{lst:supplier-cards}.

\begin{lstlisting}[language=JavaScript, caption={Rendering the Supplier Cards (\texttt{SupplierGallery.vue})},
  firstnumber=44,label={lst:supplier-cards}]
<div v-for="supplier in suppliers" :key="supplier.id" class="supplier-card">
  <div class="d-flex ga-2 align-center">
    <img v-if="supplier.image" :src="supplier.image" alt="Supplier image" />
    <v-icon v-else icon="mdi-web" />
    <h4>{{ supplier.name }}</h4>
  </div>

  <p>{{ supplier.price }}</p>

  <div>
    <b>Further links:</b>
    <p>
      <a :href="supplier.url">{{ supplier.url }}</a>
    </p>
  </div>

  <div class="text-center">
    <v-btn
      :color="
        !selectedSupplier || selectedSupplier?.id === supplier.id ? 'secondary' : '#d0d0d2'
      "
      rounded="xl"
      class="select-button text-none"
      :class="{
        selected: selectedSupplier?.id === supplier.id
      }"
      @click="selectSupplier(supplier)"
      flat
    >
      <template #prepend>
        <v-icon icon="$plus" size="small" />
      </template>
      Select Supplier
    </v-btn>
  </div>
</div>
\end{lstlisting}

Each supplier is represented by a card with details such as name, price, and a link to the supplier’s website. If an
image is available, it is displayed; otherwise, a default icon is shown. The "Select Supplier" button dynamically
changes its color and style depending on whether the supplier is selected or not, providing a clear visual indication
of the active supplier.